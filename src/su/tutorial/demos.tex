The directiory \$CWPROOT/src/demos   contains a number of subdirectories containing
ready to run shell scripts to demostrate SU programs. This collection is nowhere
as complete as it should be, and is continually changing and being updated.

\begin{itemize}
\item BC\_Examples
\item Block:  Demo for the program "block" in the free package UNCERT
\item 3D\_Model\_Building:
	\begin{itemize}
	\item Tetramod:  Demo for ``tetra,'' the seismic wavespeed model building code.
	\end{itemize}
\item Datuming: Examples of datuming
	\begin{itemize}
	\item Finite\_Difference:   Traditional Finite difference Datuming
	\item Kirchhoff\_Datuming:   Kirchhoff datuming
		\begin{itemize}
		\item Migration\_with\_topography:  Migration with topographics effects
		\end{itemize}
	\end{itemize}
\item Deconvolution:  Deconvolution demos
	\begin{itemize}
	\item Wiener\_Levinson:  prediction error filtering 
	\item FX:   frequency-wavenumber
	\end{itemize}
\item Dip\_Moveout:   Dip Moveout
	\begin{itemize}
	\item Ti:   Dip moveout in transversely isotropic media
	\end{itemize}
\item Diversity\_Stacking:
\item Filtering:   Filtering demos
	\begin{itemize}
	\item Subfilt:  Butterworth filter
	\item Sudipfilt:  Dip (slope) filtering
	\item Sufilter: Polygonal zero-phase frequency filtering
	\item Sugabor: time-frequency filtering via the Gabor transform
	\item Suk1k2filter: Wavenumber filtering (box)
	\item Sukfilter:  Wavenumber filtering (annular)
	\item Sumedian:  Median filtering
	\end{itemize}
\item Making\_Data:  Making synthetic seismic profiles
	\begin{itemize}
	\item CommonOffset: Common-offset and Zero-offset datasets
	\item ShotGathers: Common-shot datasets
	\end{itemize}
\item Migration\_Inversion:   Migration (Inversion)
	\item 3D  Migration or inversion of 3D datasets
		\begin{itemize}
		\item Suinvco3d:   Common offset migration in v(z) media
		\end{itemize}
	\item Finite\_Difference:   traditional Claerbout FD migration 
	\item Fourier\_Finite\_Difference: Fourier FD migration
	\item Gazdag: traditional Gazdag phaseshift migration
	\item Kirchhoff:  Kirchhoff migration
		\begin{itemize}
		\item Sukdmig2d\_common\_offset:   Common-offset migration
		\item Sukdmig2d\_common\_shot:   Common-shot migration
		\end{itemize}
	\item Phase\_Shift\_Plus\_Interpolation:   PSPI migration
	\item Split\_Step:  Split step migration
	\item Stolt:  Stolt migration
	\item Suinvvxzco: Inversion in v(x,z) media for common-offset data sets
	\item Sumigpsti:  Phaseshift Migration in Transversely Isotropic media
\item NMO: Traditional Normal Moveout correction
\item Offset\_Continuation:  Offset continuation to smaller offsets
\item Picking: Trace Picking
	\begin{itemize}
	\item Supickamp:   Automated picking
	\end{itemize}
\item Plotting:  Plotting demos
	\begin{itemize}
	\item CurveOnImage: Draw a curve on an image plot
	\end{itemize}
\item Refraction:   Refraction applications
	\begin{itemize}
	\item GRM:  Generalized Reciprocal Method for a single layer
	\end{itemize}
\item Residual\_Moveout: Velocity analysis by residual moveout analysis
\item SUManual:   Shell scripts used to make plots in the SU User's Manual
	\begin{itemize}
	\item Plotting: Plotting demos
		\begin{itemize}
		\item Psmerge: Merge PostScript files with {\bf psmerge\/}
		\item Xmovie: Make X-windows movies with {\bf suxmovie\/}
		\end{itemize}
	\end{itemize}
\item Selecting\_Traces:   Selecting traces by header values for specific operations
\item Sorting\_Traces:  Sorting traces with {\bf susort\/}
	\begin{itemize}
	\item Demo:  Sorting traces demo
	\item Tutorial:  Sorting traces tutorial
	\end{itemize}
\item Statics:  Static time shift corrections
	\begin{itemize}
	\item Residual\_Refraction\_Statics:  applying residual refraction static corrections
	\item Residual\_Statics: residual source-receiver statics
	\end{itemize}
\item Synthetic: Making synthetic seismograms
	\begin{itemize}
	\item Finite\_Difference: finite difference 2D modeling
	\item Impulse:  impulse testpatterns
	\item Kirchhoff:  Kirchhoff modeling
	\item Reflectivity:  Reflectivity modeling
	\item Tetra:  Tetrahedrized models
	\item Tri: Triangulated models
	\item Wkbj: Upwind finite differencing of the eikonal equation
	\item CSHOT:  see the demos in \$CWPROOT/src/Fortran/CSHOT
	\end{itemize}
\item Tau\_P:  Tau-p or slant stack filtering
	\begin{itemize}
	\item Suharlan:  Bill Harlan's slant stack noise suppression algorithm
	\item Sutaup: Traditional Tau-p transform
	\end{itemize}
\item Time\_Freq\_Analysis:  operations in the time-frequency domain
\item Utilities: Miscellaneous (and experimental) utilities
	\begin{itemize}
	\item Sucommand:   Demo for program {\bf sucommand\/}
	\end{itemize}
\item Velocity\_Analysis: NMO-based velocity analysis demo
\item Velocity\_Profiles: Make velocity profiles
	\begin{itemize}
	\item Makevel:  make velocity profiles with {\bf makevel\/}
	\item Unif2:  uniformly sampled 2D velocity profiles with {\bf unif2\/}
	\item Unisam2: uniformly sampled 2D velocity profiles with {\bf unisam2\/}
	\item Vel2stiff:  convert velocity profiles to stiffness profiles {\bf vel2stiff}
	\end{itemize}
\item Vibroseis\_Sweeps: Synthetic vibroseis sweeps with {\bf suvibro}
\end{itemize}

The beginner is encouraged to run the demos in the following order:

The Making\_Data demo shows the basics of making synthetic data
shot gathers and common offset sections using susynlv.  Particular
attention is paid to illustrating good display labeling.

The Filtering/Sufilter demo illustrates some real data processing to
eliminate ground roll and first arrivals.  The demos in the
Filtering subdirectories give instructions for accessing the data
from the CWP ftp site.

The Deconvolution demo uses simple synthetic spike traces to
illustrate both dereverberation and spiking decon using supef and
other tools.  The demos include commands for systematically
examining the effect of the filter parameters using loops.

The Sorting\_Traces Tutorial is an interactive script that
reinforces some of the basic UNIX and {\small\sf SU} lore discussed in this
document.  The interactivity is limited to allowing you to set the
pace.  Such tutorials quickly get annoying, but we felt that one
such was needed to cover some issues that didn't fit into our
standard demo format.  There is also a standard, but less complete,
demo on this topic.

The next step is to activate the Selecting\_Traces Demo.  Then
proceed to the NMO Demo.  Beyond that, visit the Demo directories
that interest you.  The {\bf demos} directory tree is still under
active development---please let us know if the demos are helpful and
how they can be improved.

As the user becomes more familiar with the logic of SU, then he or she may
feel free to run any of the other demos either in their original form,
or the user may modify the shell scripts to his or her own ends.

